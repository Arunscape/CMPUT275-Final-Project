%preamble
\documentclass{article}

%sets length of spacing between paragraphs
\setlength{\parskip}{1em}
% \usepackage{url}
\usepackage{hyperref}

\usepackage{graphicx}
\graphicspath{ {images/} }

% gets rid of section numberi
\makeatletter
\renewcommand{\@seccntformat}[1]{}
\makeatother

  \synctex=1

%temporary
% \usepackage{blindtext}

% %\usepackage[style=ieee]{biblatex}
%
% %IEEE style bibliography
% \bibliographystyle{IEEEtran}

%title page
\title{CMPUT 275 Project Proposal\\
\vspace{.25cm}\large Chess \vspace{-.5cm}}
\author{\LARGE Arun Woosaree  \& Tamara Bojovic}
\date{\today}
%actual document
\begin{document}
  \maketitle %insert titlepage here
  %Introduction

  \section{Description}
    The project is an improvement on our CMPUT 274 final project, which was a
    player vs. player Arduino chess game. We plan to first import our project
    into python, making use of the object oriented features of python, that we
    did not use in the old version. For example, each chess piece will inherit
    from a generic chess piece class, and have their own moves. Instead of the
    Arduino, we will use pygame as our GUI. We will improve on the algorithms
    from our original game, using new algorithmic knowledge gained in 275.
    (particularly check and checkmate tests, which were inefficient). Once a
    player vs. player implementation is up and running, we will then focus on
    creating a basic computer opponent. This will involve a move generator, and
    an evaluator function that determines scores for different board positions.
    We would use our knowledge of search trees, to implement the computer
    player’s moves.

    \subsection{Python classes}
    Will be used to store information about the chess pieces.
    Each chess piece can inherit from the the chess piece class, and then have
    their own moves. For example, we could define the catsle and bishop class, and then
    the queen could inherit from those classes.

    \subsection{Graph}
    A tree can be used for move generation, from which we can detect valid moves
    in a more efficient was compared to checking every single square.

    \subsection{Basic computer player}
    Once we have valid move generation down, we can maybe get a rudimentary
    computer player working to play chess against

    \subsection{Arduino}
    If time permits, we can use server communication to let the player
    interact with a joystick, and pyserial to let a computer act as a sever for processing moves




  \section{Timeline}

  \begin{itemize}
    \item March 16: Finish making this proposal
    \item March 21: Work on Assignment 2
    \item March 23: work some more on Assignment 2
    \item March 26: Refine our plans and see what's really possible
    \item March 28: Work on chess pieces and move generation
    \item.March 30: Get it working with the Arduino
    \item April 02: Sart work on computer logic
    \item April 04: idk
    \item Finishing touches
    \item present
  \end{itemize}



\end{document}
