%preamble
\documentclass{article}

%sets length of spacing between paragraphs
\setlength{\parskip}{1em}
% \usepackage{url}
% \usepackage{hyperref}

\usepackage{graphicx}
\graphicspath{ {images/} }

% gets rid of section numberi
\makeatletter
\renewcommand{\@seccntformat}[1]{}
\makeatother

  \synctex=1

%temporary
% \usepackage{blindtext}

% %\usepackage[style=ieee]{biblatex}
%
% %IEEE style bibliography
% \bibliographystyle{IEEEtran}

%title page
\title{CMPUT 275 Project Proposal\\
\vspace{.25cm}\large Chess \vspace{-.5cm}}
\author{\LARGE Arun Woosaree  \& Tamara Bojovic}
\date{\today}
%actual document
\begin{document}
  \maketitle %insert titlepage here
  %Introduction

  \section{Introduction}
  The project is an improvement on our CMPUT 274 final project, which was a
  player vs. player Arduino chess game. We plan to first import our project into
  python, making use of the object oriented features of python, that we did not
  use in the old version. For example, each chess piece will inherit from a
  generic chess piece class, and have their own moves. Instead of the Arduino,
  we will use pygame as our GUI. We will improve on the algorithms from our
  original game, using new algorithmic knowledge gained in 275. (particularly
  check and checkmate tests, which were inefficient). Once a player vs. player
  implementation is up and running, we will then focus on creating a basic
  computer opponent. This will involve a move generator, and an evaluator
  function that determines scores for different board positions. We would use
  our knowledge of search trees, to implement the computer player’s moves.


  \section{Timeline}

  \begin{enumerate}
    \item \textbf{March 22}\\ Import our old project into python. Create classes
    for the pieces, set up the board representation and use our old functions
    within the new classes. Demo: Fully working player vs. player chess game in
    pygame

    \item \textbf{March 25}\\ Improve the existing algorithms to be more
    efficient, with our new knowledge of search trees and graphs. Focus on the
    very inefficient check/checkmate detection function Demo: Player vs. player
    chess game with improved algorithms
    \newpage
    \item \textbf{April 2}\\ Create a move generator, that generates all the
    possible moves from a certain board position. Old functions that checked for
    valid moves can now utilize this. Computer player will select a move
    randomly, no optimization or intelligence Demo: Ability to play against a
    very dumb computer opponent

    \item \textbf{April 6}\\ Create a simple evaluation function, that
    calculates the best move just based on the pieces on the board. More
    important pieces will have a higher score. Demo: Computer opponent has some
    method to their madness, but cannot “look ahead” to pick the best move

    \item \textbf{April 10}\\ Increase the depth of the search to 3, and use the
    minimax algorithm to select the best move. Improve the evaluation function
    by adding more bonuses to scores, such as a protected king, control of the
    center of the board adding a bonus Demo: A fully working chess game with a
    basic computer opponent.
\end{enumerate}



\end{document}
